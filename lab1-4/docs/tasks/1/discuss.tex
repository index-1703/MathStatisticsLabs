\subsection{Гистограммы}
Полученные результаты работы говорят о том, что при увеличении размеров выборок, гистограммы все ближе к графику плотности вероятности того закона, по которому были сгенерированы элементы выборок. Верно и обратно: чем меньше выборка, тем хуже по ней можно определить закон, по которой эта выборка генерировалась. 

Также одним из ключевых выводов является тот факт, что по
маленькому размеру выборки (n = 10) очень трудно отличить гистограммы, а,
следовательно, и определить закон, по которой генерировалась выборка. Действительно, гистограмма выборки, построенной по распределению Пуассона при n = 10, могла бы с тем же успехом описывать график равномерного распределения (если не учитывать один единственный всплеск гистограммы, который вообще мог остаться незамеченным при более
широких интервалах боксов гистограммы).

При выборках n = 1000 видно, что гистограммы уже достаточно
неплохо приближаются к графикам плотностей соответствующих законов
распределения: в равномерном распределении отклонения гистограммы от
графика незначительны, а в нормальном распределении уже наблюдаются
«хвосты», которые позволяют отличить треугольное распределение от
нормального.